
%%%%%%%%%%%%%%%%%%%%%%%%%%%%%%%%%%%%%%%%%
% Medium Length Professional CV
% LaTeX Template
% Version 3.0 (December 17, 2022)
%
% This template originates from:
% https://www.LaTeXTemplates.com
%
% Author:
% Vel (vel@latextemplates.com)
%
% Original author:
% Trey Hunner (http://www.treyhunner.com/)
%
% License:
% CC BY-NC-SA 4.0 (https://creativecommons.org/licenses/by-nc-sa/4.0/)
%
%%%%%%%%%%%%%%%%%%%%%%%%%%%%%%%%%%%%%%%%%

%----------------------------------------------------------------------------------------
%	PACKAGES AND OTHER DOCUMENT CONFIGURATIONS
%----------------------------------------------------------------------------------------

\documentclass[
  %a4paper, % Uncomment for A4 paper size (default is US letter)
  10pt, % Default font size, can use 10pt, 11pt or 12pt
]{resume}

\usepackage{times}
\usepackage{hyperref}
\hyphenpenalty=10000
\exhyphenpenalty=10000
\frenchspacing

%------------------------------------------------

\name{Abhay Shukla} % Your name to appear at the top

% You can use the \address command up to 3 times for 3 different addresses or pieces of contact information
% Any new lines you use in the \address commands will be converted to symbols, so each address will appear as a single line.
\address{01shuklabhay@gmail.com $\vert$ \underline{\href{https://www.linkedin.com/in/shuklabhay/}{https://www.linkedin.com/in/shuklabhay/}} $\vert$ \underline{\href{https://github.com/shuklabhay}{https://github.com/shuklabhay}} $\vert$ \underline{\href{https://shuklabhay.github.io}{https://shuklabhay.github.io}}}% Contact information

%----------------------------------------------------------------------------------------

\begin{document}
\sloppy

%----------------------------------------------------------------------------------------
%	WORK EXPERIENCE SECTION
%----------------------------------------------------------------------------------------

\begin{rSection}{Experience}


  \begin{rSubsection}{Stanford Center for Biomedical Informatics Research}{November 2024 - Present}{Gevaert Lab Student Researcher}{California}
    
    \item Developed physics-based neural networks for anatomically constrained image augmentations (e.g., tissue compression, respiratory/cardiac phase variations), improving cancer detection accuracy by 4–20\%.
    
    \item Implemented computer vision models (CNNs, Vision Transformers) for precise breast region segmentation.
    
    \item Designed Python data processing pipelines for efficient handling of large-scale biomedical datasets on multi-node systems.
    
  \end{rSubsection}
        
  \begin{rSubsection}{FRC Team 604: Quixilver Robotics}{June 2022 - Present}{Controls/Software Lead (10)}{California}
    
    \item Implemented real-time computer vision systems for autonomous robot navigation and object detection.
    
    \item Designed strategic robotic mechanisms and integrated multi-modal sensors for maximum performance and reliability.
    
    \item Developed competition data collection infrastructure and visualization tools analyzing 1000+ competition matches with 200+ users.
    
    \item Achieved top 0.1\% international ranking (12/10,000+ teams) through integrated hardware-software optimization.
    
  \end{rSubsection}
        
  \begin{rSubsection}{UCLA COSMOS (Neurobiology and AI Cohort)}{July 2024 - August 2024}{Student Researcher}{California}
    
    \item Developed neural networks to model rat hippocampus activity, perform image geolocation, and recognized handwritten characters.
    
    \item Applied neurobiological priors to computational modeling techniques using Torch and TensorFlow.
    
  \end{rSubsection}
        
	
\end{rSection}

%----------------------------------------------------------------------------------------
%	PROJECTS SECTION
%----------------------------------------------------------------------------------------

\begin{rSection}{Projects}

  
    \begin{rSubsection}{Voquel - \textit{\underline{\href{https://www.youtube.com/@translateanyaudio}{Algorithm Demos}}}}{}{}{}
        
      \item Developed a full-stack AI translator/dubber preserving rhythm, emotion, and artistic intent across languages.
          
      \item Created efficient GPU-accelerated DSP pipeline optimised for 4 GB VRAM consumer GPUs.
          
      \item Received Honorable Mention at the 2025 Synopsys Science Fair (top 10\% out of \textasciitilde{}1000 competitors); 30+ hours of Voquel-translated content watched on YouTube.
          
    \end{rSubsection}
        
    \begin{rSubsection}{Tessera}{}{}{}
        
      \item Developed an end-to-end conversational voice agent for auditory rehabilitation through adaptive listening exercises.
          
      \item Engineered context-aware LLM-managed session state, enabling personalised progression in a lightweight (30 MB) app ready for clinical use.
          
    \end{rSubsection}
        
    \begin{rSubsection}{PercGAN - \textit{\underline{\href{https://github.com/shuklabhay/percgan}{Independent Research Project}}}}{}{}{}
        
      \item Developed a generative adversarial network addressing critical research gaps in lightweight high-quality stereo audio synthesis.
          
      \item Implemented image-like audio representations and efficient model training techniques.
          
      \item Achieved an 85\% quality improvement and a 25× training-time reduction compared with industry benchmarks.
          
    \end{rSubsection}
        
    \begin{rSubsection}{SporeStrike - \textit{\underline{\href{https://shuklabhay.github.io/static/projects/sporestrike/FlexFactor_SporeStrike_pitch.pdf}{Entrepreneurship Project}}}}{}{}{}
        
      \item Designed computational models for drone-based fungal infection treatment system, created prototype 3D printed components.
          
      \item Presented the project to a civil and aerospace engineering panel; won first place out of 260 competitors at the 2024 FlexFactor Entrepreneurship Championships.
          
    \end{rSubsection}
        

\end{rSection}
    
%----------------------------------------------------------------------------------------
%	EDUCATION SECTION
%----------------------------------------------------------------------------------------

\begin{rSection}{Education}
	
  
  \textbf{Leland High School} \hfill \textit{4.00 UW A-G} \\
  Junior (Expected Graduation 2026) \hfill \textit{San Jose, CA}
	
\end{rSection}

%----------------------------------------------------------------------------------------
    %	SKILLS SECTION
%----------------------------------------------------------------------------------------

\begin{rSection}{Skills}
 
  \begin{tabular}{@{} >{\bfseries}l @{\hspace{6ex}} l @{}}
		Technical Fields & AI/ML, Data Visualization and Analysis, Robotics, Signal Processing, 3D Printing, Web Development \\
	\end{tabular}

\end{rSection}

%----------------------------------------------------------------------------------------
    % HONORS & AWARDS SECTION
%----------------------------------------------------------------------------------------

\begin{rSection}{Honors \& Awards}

  \begin{itemize}
      \setlength\itemsep{-0.7em} % Adjust the space between items
        
      \item 2025 Synopsys Science Fair Honorable Mention (Top 10\%, 975+ Participants) \hfill 2025
              
      \item WCP CADathon/Robot Design Challenge Finalist (Top 1\%, 1000+ participants) \hfill 2024
              
      \item FRC604: World Championship Milstein Division Winner (12/3500 Internationally, 4/300 in CA) \hfill 2024
              
      \item OneHacks III Hackathon: Third Place (3/120 Internationally) \hfill 2023
              


    \end{itemize}

\end{rSection}

\end{document}
