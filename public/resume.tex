
%%%%%%%%%%%%%%%%%%%%%%%%%%%%%%%%%%%%%%%%%
% Medium Length Professional CV
% LaTeX Template
% Version 3.0 (December 17, 2022)
%
% This template originates from:
% https://www.LaTeXTemplates.com
%
% Author:
% Vel (vel@latextemplates.com)
%
% Original author:
% Trey Hunner (http://www.treyhunner.com/)
%
% License:
% CC BY-NC-SA 4.0 (https://creativecommons.org/licenses/by-nc-sa/4.0/)
%
%%%%%%%%%%%%%%%%%%%%%%%%%%%%%%%%%%%%%%%%%

%----------------------------------------------------------------------------------------
%	PACKAGES AND OTHER DOCUMENT CONFIGURATIONS
%----------------------------------------------------------------------------------------

\documentclass[
  %a4paper, % Uncomment for A4 paper size (default is US letter)
  10pt, % Default font size, can use 10pt, 11pt or 12pt
]{resume}

\usepackage{times}
\usepackage{hyperref}
\hyphenpenalty=10000
\exhyphenpenalty=10000
\frenchspacing

%------------------------------------------------

\name{Abhay Shukla} % Your name to appear at the top

% You can use the \address command up to 3 times for 3 different addresses or pieces of contact information
% Any new lines you use in the \address commands will be converted to symbols, so each address will appear as a single line.
\address{01shuklabhay@gmail.com $\vert$ \underline{\href{https://www.linkedin.com/in/shuklabhay/}{https://www.linkedin.com/in/shuklabhay/}} $\vert$ \underline{\href{https://github.com/shuklabhay}{https://github.com/shuklabhay}} $\vert$ \underline{\href{https://shuklabhay.github.io}{https://shuklabhay.github.io}}}% Contact information

%----------------------------------------------------------------------------------------

\begin{document}
\sloppy

%----------------------------------------------------------------------------------------
%	WORK EXPERIENCE SECTION
%----------------------------------------------------------------------------------------

\begin{rSection}{Experience}


  \begin{rSubsection}{Stanford Center for Biomedical Informatics Research}{November 2024 - July 2025}{Gevaert Lab Student Researcher}{California}
    
    \item Worked with researchers from Stanford, UPenn, and UC Davis to developed PINNs for physically constrained medical image augmentation, improving cancer detection accuracy by 4–20\%.
    
    \item Implemented computer vision models (CNNs, Vision Transformers) for precise breast region segmentation, Designed Python data processing pipelines for efficient handling of large-scale biomedical datasets on multi-node systems.
    
  \end{rSubsection}
        
  \begin{rSubsection}{FRC Team 604: Quixilver Robotics}{June 2022 - Present}{Controls/Software Lead (10)}{California}
    
    \item Implemented real-time localization and piece detection computer vision systems for three large-scale, 125lb competition robots.
    
    \item Designed and led the fabrication of numerous robot components, applying advanced techniques in both additive manufacturing (TPU, PETG, PLA 3D printing) and sheet metal construction.
    
    \item Brought the team to top 0.1\% international ranking (5/3,500+ teams) in 2024.
    
    \item Developed data collection infrastructure powering 200+ users to analyze 1000+ competition matches.
    
  \end{rSubsection}
        
  \begin{rSubsection}{UCLA COSMOS (Neurobiology and AI Cohort)}{July 2024 - August 2024}{Student Researcher}{California}
    
    \item Created a digital twin to model rodent pathfinding capabilities, accurately replicating true navigational behaviors and cell activations.
    
    \item Developed a UCLA campus image geolocation model by organizing campus-wide video data collection, creating custom artifact-free frame extraction, and using the resulting 1,500+ curated images to fine tune AlexNet.
    
  \end{rSubsection}
        
	
\end{rSection}

%----------------------------------------------------------------------------------------
%	PROJECTS SECTION
%----------------------------------------------------------------------------------------

\begin{rSection}{Projects}

  
    \begin{rSubsection}{Voquel - \textit{\underline{\href{https://www.youtube.com/@translateanyaudio}{Algorithm Demos}}}}{}{}{}
        
      \item Full stack LLM-enhanced research project for accessible audio translation with a focus on preserving rhythm, emotion, and artistic intent.
          
      \item Content translated by Voquel has reached almost 1 Million viewers over 70 different videos.
          
      \item Received Honorable Mention at the 2025 Synopsys Science Fair (top 10\% out of \textasciitilde{}1000 competitors); recognized by the City of San Jose for language preservation efforts.
          
    \end{rSubsection}
        
    \begin{rSubsection}{PercGAN - \textit{\underline{\href{https://github.com/shuklabhay/percgan}{Independent Research Project}}}}{}{}{}
        
      \item Developed a lightweight generative network for high-fidelity stereo percussion generation presenting unique audio representation and enhanced StyleGAN architecture.
          
      \item Achieved an 85\% quality improvement and 25× training-time reduction compared to WaveGAN, MelGAN, etc..
          
    \end{rSubsection}
        
    \begin{rSubsection}{Tessera - \textit{\underline{\href{https://shuklabhay.github.io/tessera/}{Landing Page}}}}{}{}{}
        
      \item Developed an end-to-end conversational voice agent delivering adaptive listening exercises for auditory rehabilitation.
          
      \item Engineered context-aware LLM-managed session state, enabling personalized progression in a 30 MB application optimized for clinical deployment.
          
    \end{rSubsection}
        
    \begin{rSubsection}{SporeStrike - \textit{\underline{\href{https://shuklabhay.github.io/static/projects/sporestrike/FlexFactor_SporeStrike_pitch.pdf}{Entrepreneurship Project}}}}{}{}{}
        
      \item Created aerial fungal infection treatment system with 3D-printed prototype components for targeted agricultural application.
          
      \item Won first place out of 260 teams at the 2024 FlexFactor Entrepreneurship Championships, presenting to civil and aerospace engineering panels.
          
    \end{rSubsection}
        

\end{rSection}
    
%----------------------------------------------------------------------------------------
%	EDUCATION SECTION
%----------------------------------------------------------------------------------------

\begin{rSection}{Education}
	
  
  \textbf{Leland High School} \hfill \textit{4.00 UW A-G} \\
  Junior (Expected Graduation 2026) \hfill \textit{San Jose, CA}
	
\end{rSection}

%----------------------------------------------------------------------------------------
    %	SKILLS SECTION
%----------------------------------------------------------------------------------------

\begin{rSection}{Skills}
 
  \begin{tabular}{@{} >{\bfseries}l @{\hspace{6ex}} l @{}}
		Technical Fields & AI/ML, Data Visualization and Analysis, Robotics, Signal Processing, 3D Printing, Web Development \\
	\end{tabular}

\end{rSection}

%----------------------------------------------------------------------------------------
    % HONORS & AWARDS SECTION
%----------------------------------------------------------------------------------------

\begin{rSection}{Honors \& Awards}

  \begin{itemize}
      \setlength\itemsep{-0.7em} % Adjust the space between items
        
      \item 2025 Synopsys Science \& Engineering Fair Honorable Mention (Top 10\%, 975+ Participants) \hfill 2025
              
      \item WCP CADathon/Robot Design Challenge Finalist (Top 1\%, 1000+ participants) \hfill 2024
              
      \item FRC604: World Championship Milstein Division Winner (5/3500 Internationally) \hfill 2024
              
      \item OneHacks III Hackathon: Third Place (3/120 Internationally) \hfill 2023
              


    \end{itemize}

\end{rSection}

\end{document}
