
%%%%%%%%%%%%%%%%%%%%%%%%%%%%%%%%%%%%%%%%%
% Medium Length Professional CV
% LaTeX Template
% Version 3.0 (December 17, 2022)
%
% This template originates from:
% https://www.LaTeXTemplates.com
%
% Author:
% Vel (vel@latextemplates.com)
%
% Original author:
% Trey Hunner (http://www.treyhunner.com/)
%
% License:
% CC BY-NC-SA 4.0 (https://creativecommons.org/licenses/by-nc-sa/4.0/)
%
%%%%%%%%%%%%%%%%%%%%%%%%%%%%%%%%%%%%%%%%%

%----------------------------------------------------------------------------------------
%	PACKAGES AND OTHER DOCUMENT CONFIGURATIONS
%----------------------------------------------------------------------------------------

\documentclass[
  %a4paper, % Uncomment for A4 paper size (default is US letter)
  10pt, % Default font size, can use 10pt, 11pt or 12pt
]{resume} % Use the resume class

\usepackage{ebgaramond} % Use the EB Garamond font
\usepackage{hyperref}

%------------------------------------------------

\name{Abhay Shukla} % Your name to appear at the top

% You can use the \address command up to 3 times for 3 different addresses or pieces of contact information
% Any new lines you use in the \address commands will be converted to symbols, so each address will appear as a single line.
\address{abhayshuklavtr@gmail.com $\vert$ \underline{\href{https://www.linkedin.com/in/shuklabhay/}{https://www.linkedin.com/in/shuklabhay/}} $\vert$ \underline{\href{https://github.com/shuklabhay}{https://github.com/shuklabhay}} $\vert$ \underline{\href{https://shuklabhay.github.io}{https://shuklabhay.github.io}}}% Contact information

%----------------------------------------------------------------------------------------

\begin{document}
\sloppy

%----------------------------------------------------------------------------------------
%	WORK EXPERIENCE SECTION
%----------------------------------------------------------------------------------------

\begin{rSection}{Experience}


  \begin{rSubsection}{FRC 604: Quixilver Robotics}{2022 - Present}{Controls/Software Lead (10-11)}{California}
    
    \item Enhanced robot capabilities with computer vision and sensor integration, contributed to SOTA team software (particle filter localizer, time-optimal trajectory optimizer, annual robot control architecture). Developed team's competition data collection application Quickscout.
    
    \item Developed a progressive web app for TheBlueAlliance, modernizing the univerally used platform and providing comprehensive, real-time access to all competition and team data over 30+ years.
    
    \item Led subteams in design, manufacturing, and programming of FRC team robot while teaching new members about robot development.
    
  \end{rSubsection}
        
  \begin{rSubsection}{UCLA COSMOS (CA Summer School for Mathematics \& Science)}{2024}{Brain-Inspired Computing Cohort Member}{California}
    
    \item Conducted deep analyses of the computational principles and neurological correlations of 20+ key machine learning mechanisms (attention, visual processing, recurrent systems, reinforcement, etc), gaining a comprehensive understanding of their capabilities and limitations.
    
    \item Developed StereoSampleGAN with UCLA funding to generate high quality stereo audio samples with a 99.77\% reduction in training epoch count and and 9.56x reduction in parameter count compared to compared to monophonic audio generation models SpecGAN and WaveFlow.
    
  \end{rSubsection}
        
  \begin{rSubsection}{Bay Area STEM Academy}{2023 - Present}{Cofounder}{California}
    
    \item Taught 700+ elementary to high school students diverse STEM topics via numerous in-person and online workshops covering Robotics, Engineering, Machine Learning, and Programming Fundamentals.
    
    \item Raised \$6000+ to support local STEM outreach initiatives for underrepresented communities and children health foundations, recruited 15+ academy mentors.
    
  \end{rSubsection}
        
	
\end{rSection}

%----------------------------------------------------------------------------------------
%	PROJECTS SECTION
%----------------------------------------------------------------------------------------

\begin{rSection}{Projects}

  
  \begin{rSubsection}{StereoSampleGAN}{}{GitHub Repo: \underline{\href{https://github.com/shuklabhay/stereo-sample-gan}{https://github.com/shuklabhay/stereo-sample-gan}}}{}
       
    \item WGAN-based computationally effiicent approach for generating high-fidelity stereo audio samples. Leverages attention mechanisms, optimized loss functions, and effective signal processing. Research partially funded by UCLA and pending publication.
        
    \item New architecture overcame low-quality and monophonic limitations of existing audio generation methods with a 99.77\% reduction in training epoch count and 9.56x reduction in parameter count compared to compared to SpecGAN and WaveFlow architectures.
        
  \end{rSubsection}
      
  \begin{rSubsection}{Quickscout}{}{GitHub Organization: \underline{\href{https://github.com/frc604}{https://github.com/frc604}}}{}
       
    \item Developed scalable and flexible robot performance analysis application for multimodal FRC event data collection and visualization. New captured metrics empower informed strategic decisions at high-stakes competitions, contributing to the team's international success.
        
    \item Supports 130+ users on FRC604, data collected for 700+ team robots over 1 year. Trained 10+ team members in webdev to build app and expand application capabilities to fit annual challenges.
        
  \end{rSubsection}
      
  \begin{rSubsection}{Domotron}{}{Robot Website: \underline{\href{https://604robotics.com/2023-2024-crescendo/}{https://604robotics.com/2023-2024-crescendo/}}}{}
       
    \item Developed computer vision and physics-based shot calculations, driver control automation, and competitive autonomous routines for world championship division winning robot.
        
    \item Designed climber winch mechanism and robot vertical elevator, maximizing robot's competitive capabilities and earning the robot the Industrial Design, Innovation in Control, and Autonomous Awards.
        
  \end{rSubsection}
      
	
\end{rSection}
    
%----------------------------------------------------------------------------------------
%	EDUCATION SECTION
%----------------------------------------------------------------------------------------

\begin{rSection}{Education}
	
  
  \textbf{Leland High School} \hfill \textit{4.00 UW A-G} \\
  Junior \hfill \textit{San Jose, CA}
	
\end{rSection}

%----------------------------------------------------------------------------------------
    %	SKILLS SECTION
%----------------------------------------------------------------------------------------

\begin{rSection}{Skills}

  \begin{tabular}{@{} >{\bfseries}l @{\hspace{6ex}} l @{}}
		Technical Fields & AI/ML, Robotics, Signal Processing, CAD, 3D Printing, Webdev/Appdev \\
    Other & Digital Audio Production, Graphic Design, Video Editing \\
	\end{tabular}

\end{rSection}

%----------------------------------------------------------------------------------------
    % HONORS & AWARDS SECTION
%----------------------------------------------------------------------------------------

\begin{rSection}{Honors \& Awards}

  \begin{itemize}
      \setlength\itemsep{-0.7em} % Adjust the space between items
        
      \item FRC604: World Championship Milstein Division Winner (12/3500 Internationally + 4/300 in CA) \hfill 2024
              
      \item NextFlex FlexFactor: Entrepreneurship Competition Winner (1/260 in CA) \hfill 2024
              
      \item OneHacks III Hackathon: Third Place (3/120 Internationally) \hfill 2023
              
      \item SCU/SVUDL Invitational: PF Debate Finalist (2/140 Internationally) \hfill 2022
              


    \end{itemize}

\end{rSection}

\end{document}
